\title{Pöhinää}

Suoraa quotea SLR:stä: 

"challenge themes in this study refer to all challenges related to team member design, coordination, behavior and other team practices."

Merkittävä aihe, sillä ketterä periaate: yksilöt ja vuorovaikutus ylitse prosessien ja työkalujen.
-> Tiimihallinta on erittäin haastavaa ja vaatii erityishuomiota softaprojektin onnistumisen takaamiseksi.


Hallinnalliset haasteet jaot:

Silva,      [1]: 
- Erimielisyydet tiimissä, 
- hankaluudet uuden tiimiläisen ottamisessa, 
- siiloutuminen (\textit{offshoring}) rajatun ryhmän kanssa, 
- tiimiläisen korvaaminen, 
- tiimin käytös, 
- muutoksen vastustus (resistance to change)

Claps,      [2]:
- Muutokset tiimin rooleissa (\textit{team role}), 
- tiimin koordinointi, 
- tiimin kokemus

Gregory,    [3]:
- Tiimikäytänteet,
- Johtajuus,
- Hyvien tiimiläisten löytäminen,
- Yksilöllinen motivaatio


Lisäyksiä haasteisiin:

De Melo     [4]:
- Tiimiläisten vaihtuvuus                               (check)
- Tiimin suunnitteluvalinta \textit{team design choice} (check)

Alzoubi     [5]:
- Tiimin konfigurointi (menee suunnitteluvalintaan?)
- Kommunikaatio

Moe         [6]:
- Tiimiperehdytyksen puute,
- Yksilöllisen kohtaamisen puute


Aihealuejakoa:

Koordinaatio:
- Ketterän käyttöönotto             [3]
- Tiimin kommunikointi              [5]
-> Erimielisyydet tiimissä          [1]
-> Yksilöllisen kohtaamisen puute   [6]
-> Perehdytyksen puute              [6]

Kokoonpano:
- (Ryhmäkokoonpano)
- Siiloutuminen
- Muutokset tiimin rooleissa
- Suunnitteluvalinnat
- Vaihtuvuus / Tiimiläisen korvaaminen
- Johtajuus [3]
- Hyvien tiimiläisten löytäminen
- Ryhmien välinen koordinaatio


TODO:
- Kokoonpanokappaleen viimeistely
- Pöhinäkappaleen teko (jotta tekstissä on omaäänisyyttä)
-> Haasteiden ja ratkaisumallien vertailua
--> Jos ei ratkaisumalleja, mahdollisten mallien pohtimista
- Päivitä johdanto tutkielman mukaiseksi
- Yhteenveto
- Tiivistelmän viimeistely
- Koordinaatiokappaleen viitteiden parantelu
- Kielioppi- ja loogisuuscheck
- Palautus
