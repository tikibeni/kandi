\title{Pöhinää}

Suoraa quotea SLR:stä: 

"challenge themes in this study refer to all challenges related to team member design, coordination, behavior and other team practices."

Merkittävä aihe, sillä ketterä periaate: yksilöt ja vuorovaikutus ylitse prosessien ja työkalujen.
-> Tiimihallinta on erittäin haastavaa ja vaatii erityishuomiota softaprojektin onnistumisen takaamiseksi.


Hallinnalliset haasteet jaot:

Silva, 1: 
- Erimielisyydet tiimissä, 
- hankaluudet uuden tiimiläisen ottamisessa, 
- siiloutuminen (\textit{offshoring}) rajatun ryhmän kanssa, 
- tiimiläisen korvaaminen, 
- tiimin käytös, 
- muutoksen vastustus (resistance to change)

Claps, 2:
- Muutokset tiimin rooleissa (\textit{team role}), 
- tiimin koordinointi, 
- tiimin kokemus

Gregory, 3 (yleisesti tiimiin liittyen?):
- Tiimikäytänteet,
- Johtajuus,
- Hyvien tiimiläisten löytäminen,
- Yksilöllinen motivaatio


Lisäyksiä haasteisiin:

De Melo [4]:
- Tiimiläisten vaihtuvuus
- Tiimin suunnitteluvalinta \textit{team design choice}
-> Lue mitä tämä tarkoittaa

Alzoubi [5]:
- Tiimin konfigurointi

Moe [6]:
- Tiimiperehdytyksen puute,
- Yksilöllisen kohtaamisen puute


Aihealuejakoa:

Koordinaatio:
- Ketterän käyttöönotto
- Tiimin kommunikointi
-> Erimielisyydet tiimissä [1]

Tiimin kokoonpano
- Siiloutuminen
- Muutokset tiimin rooleissa
- Team design choice
- Vaihtuvuus / Tiimiläisen korvaaminen
- Johtajuus
- Hyvien tiimiläisten löytäminen
- 

Yksilömuuttujat:
- Yksilöllinen motivaatio
- Käytökselliset seikat
- 


TODO:
- Parempi kappalejako
- Lisää koordinaatiokappaleeseen jokin bängeri kaavio ":D"
- Kaavio eri tuotantomallien tuotantotavoista
- Sivu, jossa lueteltuna ohjaaja ja valvoja
- Tiivistelmä + luokitelma
