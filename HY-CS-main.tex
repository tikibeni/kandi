%% Follow comments to support use.

%%%%%%%%%%%%%%%%%%%%%%%%%%%%%%%%%%%%%%%%%%%%%%%%%%%%%%%%%
%% STEP 1: Choose options for MSc / BSc / seminar layout and your bibliographic style
%%%%%%%%%%%%%%%%%%%%%%%%%%%%%%%%%%%%%%%%%%%%%%%%%%%%%%%%%

\documentclass[finnish,twoside,censored,tkt]{HYthesisML} 

% If wanted, open new chapters only at right page.
% By default, "openany".
%\PassOptionsToClass{openright,twoside,a4paper}{report}
\PassOptionsToClass{openany,twoside,a4paper}{report}

\usepackage{csquotes}
%%%%%%%%%%%%%%%%%%%%%%%%%%%%%%%%%%%%%%%%%%%%%%%%%%%%%%%%%
%% REFERENCES
%% Some notes on bibliography usage and options:
%% natbib -> you can use, e.g., \citep{} or \parencite{} for (Einstein, 1905); with APA \cite -> Einstein, 1905 without ()
%% maxcitenames=2 -> only 2 author names in text citations, if more -> et al. is used
%% maxbibnames=99 as no great need to suppress the biliography list in a thesis
%% for more information see biblatex package documentation, e.g., from https://ctan.org/pkg/biblatex 

%% Reference style: select one 
%% for APA = Harvard style = authoryear -> (Einstein, 1905) use:
%\usepackage[style=authoryear,bibstyle=authoryear,backend=biber,natbib=true,maxnames=99,maxcitenames=2,uniquelist=minyear,giveninits=true,uniquename=mininit]{biblatex}
%% for numeric = Vancouver style -> [1] use:
\usepackage[style=numeric,bibstyle=numeric,backend=biber,natbib=true,maxbibnames=99,giveninits=true,uniquename=init]{biblatex}
%% for alpahbetic -> [Ein05] use:
%\usepackage[style=alphabetic,bibstyle=alphabetic,backend=biber,natbib=true,maxbibnames=99,giveninits=true,uniquename=init]{biblatex}
%

\addbibresource{bibliography.bib}
% in case you want the final delimiter between authors & -> (Einstein & Zweistein, 1905) 
% \renewcommand{\finalnamedelim}{ \& }
% List the authors in the Bibilipgraphy as Lastname F, Familyname G,
\DeclareNameAlias{sortname}{family-given}
% remove the punctuation between author names in Bibliography 
%\renewcommand{\revsdnamepunct}{ }


%% Block of definitions for fonts and packages for picture management.
%% In some systems, the figure packages may not be happy together.
%% Choose the ones you need.

%\usepackage[utf8]{inputenc} % For UTF8 support, in some systems. Use UTF8 when saving your file.

\usepackage{lmodern}         % Font package, again in some systems.
\usepackage{textcomp}        % Package for special symbols
\usepackage[pdftex]{color, graphicx} % For pdf output and jpg/png graphics
\usepackage{epsfig}
\usepackage{subfigure}
\usepackage[pdftex, plainpages=false]{hyperref} % For hyperlinks and pdf metadata
\usepackage{fancyhdr}        % For nicer page headers
\usepackage{tikz}            % For making vector graphics (hard to learn but powerful)
%\usepackage{wrapfig}        % For nice text-wrapping figures (use at own discretion)
\usepackage{amsmath, amssymb} % For better math

\singlespacing               %line spacing options; normally use single

\fussy
%\sloppy                      % sloppy and fussy commands can be used to avoid overlong text lines
% if you want to see which lines are too long or have too little stuff, comment out the following lines
% \overfullrule=1mm
% to see more info in the detailed log about under/overfull boxes...
% \showboxbreadth=50 
% \showboxdepth=50



%%%%%%%%%%%%%%%%%%%%%%%%%%%%%%%%%%%%%%%%%%%%%%%%%%%%%%%%%
%% STEP 2:
%%%%%%%%%%%%%%%%%%%%%%%%%%%%%%%%%%%%%%%%%%%%%%%%%%%%%%%%%
%% Set up personal information for the title page and the abstract form.
\title{Ketterän ohjelmistokehitysryhmän hallinnalliset haasteet}
\author{Benjamin Blinnikka}
\date{\today}

% Set supervisors, use the titles according to the thesis language
\supervisors{FM Nea Pirttinen}
\examiners{TkT Emilia Oikarinen}

\keywords{ketterä, ohjelmistokehitys, kehitysryhmä, hallinta, haasteet}
\additionalinformation{\translate{\track}}

%% Provide classification terms, to appear on the abstract page.
\classification{\protect{\ \\
    Software and its engineering $\rightarrow$ 
    Software creation and management \\ $\rightarrow$ 
    Software development process management $\rightarrow$  
    Software development methods \\ $\rightarrow$ 
    Agile software development
}}

%% If you want to quote someone special. You can comment this line out and there will be nothing on the document.
%\quoting{Bachelor's degrees make pretty good placemats if you get them laminated.}{Jeph Jacques}


%% OPTIONAL STEP: Set up properties and metadata for the pdf file that pdfLaTeX makes.
\hypersetup{
    unicode=true,                                   % to show non-Latin characters in Acrobat’s bookmarks
    pdftoolbar=true,                                % show Acrobat’s toolbar?
    pdfmenubar=true,                                % show Acrobat’s menu?
    pdffitwindow=false,                             % window fit to page when opened
    pdfstartview={FitH},                            % fits the width of the page to the window
    pdftitle={Ketterän ohjelmistokehitysryhmän hallinnalliset haasteet},            % title
    pdfauthor={Benjamin Blinnikka},                                                 % author
    pdfsubject={Ketterän ohjelmistokehitysryhmän hallinnalliset haasteet},          % subject of the document
    pdfcreator={Benjamin Blinnikka},                % creator of the document
    pdfproducer={pdfLaTeX},                         % producer of the document
    pdfkeywords={ketterä} {ohjelmistokehitys} {kehitysryhmä} {hallinta} {haasteet}, % list of keywords for
    pdfnewwindow=true,                              % links in new window
    colorlinks=true,                                % false: boxed links; true: colored links
    linkcolor=black,                                % color of internal links
    citecolor=black,                                % color of links to bibliography
    filecolor=magenta,                              % color of file links
    urlcolor=cyan                                   % color of external links
}

%%-----------------------------------------------------------------------------------

\begin{document}

% Generate title page.
\maketitle

%%%%%%%%%%%%%%%%%%%%%%%%%%%%%%%%%%%%%%%%%%%%%%%%%%%%%%%%%
%% STEP 3:
%%%%%%%%%%%%%%%%%%%%%%%%%%%%%%%%%%%%%%%%%%%%%%%%%%%%%%%%%
%% Write your abstract in the separate file, to be positioned here.
%% You can make several abstract pages (if you want it in different languages),
%% in which case you should also define the language of the abstract,
%% as below.

\begin{abstract}

Ketterä ohjelmistokehitys on kehitetty korjaamaan perinteisten tuotantomenetelmien ja -mallien ongelmakohtia. Vaikka ketterä manifesti korjasi ketterää edeltävän iteratiivisen mallin aatteelliset ongelmat, ketterä sisältää omat haastealueensa, joita on haluttu viime vuosina kartoittaa ketterän kehittämistä varten. Merkittävimmäksi haastealueeksi ilmaantuvuuden perusteella on todettu ketterää käyttävän ryhmän hallinta (\textit{team management}). 

Tämä kirjallisuuskatsauksellinen tutkielma keskittyy ja vastaa seuraavanlaisiin tutkimuskysymyksiin: 1. Mitä hallinnallisia haasteita liittyy ketterään ryhmään? 2. Mitä ratkaisumalleja on olemassa ketterän ryhmän hallinnallisiin haasteisiin?

Yleisimmiksi haastealueiksi muodostuivat koordinaatiiviset ja kokoonpanolliset haasteet. Koordinaatiohaasteita ovat ketterän käyttöönotto, kommunikointi, useamman ryhmän välinen yhteistoiminta sekä ryhmäorientaatio. Kokoonpanollisia haasteita ovat ryhmän suunnitteluvalinnat sekä ryhmäläisten vaihtuvuus. Ratkaisuina on esitetty konsultoinnin hakeminen ammattilaiselta, ryhmälähtöinen ongelmanratkaisu, epävirallisten kommunikointitapojen ja lähitapaamisten suosiminen, virallisen kommunikoinnin lisääminen, selkeä roolijako, paikallisryhmien muodostaminen, jatkuvan kommunikoinnin työkulttuurin suosiminen, luottamuksen lisääminen, lisäkommunikointi asiakkaan kanssa, ketterän ryhmän kyky mukautua tilanteisiin ja sopiva ohjaus organisaatiojohdon toimesta.

Tutkielma toteaa lisäksi tarpeen lisätutkimuksille ratkaisumallien osalta ratkaisuja tarjoavien lähteiden ollessa vähäisiä.

\end{abstract}


% Place ToC
%\newpage
\mytableofcontents

\mainmatter

%%%%%%%%%%%%%%%%%%%%%%%%%%%%%%%%%%%%%%%%%%%%%%%%%%%%%%%%%
%% STEP 4: Write the thesis.
%%%%%%%%%%%%%%%%%%%%%%%%%%%%%%%%%%%%%%%%%%%%%%%%%%%%%%%%%
%% Your actual text starts here. You shouldn't mess with the code above the line except
%% to change the parameters. Removing the abstract and ToC commands will mess up stuff.
%%
%% Command \include{file} includes the file of name file.tex.
%% A new page will be created at every \include command, 
%% which makes it appropriate to use it for large entities such as book chapters. Cannot be nested.
%% It is useful for a big project, as changing one of the include targets 
%% won't force the regeneration of the outputs of all the rest.
%% Alternatively, \input is a more lower level macro 
%% which simply inputs the content of the given file like it was copy&pasted there manually.

\chapter{Johdanto\label{intro}}

Ketterä ohjelmistokehitys (jatkossa \textit{ketterä}) on suosittu kehitysaate nykypäivän ohjelmistokehityksessä, minkä avulla ohjelmistoja voidaan rakentaa ja muokata tehokkaasti asiakkaan tarpeiden ympärille. Ketterän suosio perustuu periaatteisiin, joiden myötä ohjelmistoprojektien onnistuvuusprosentin ja asiakastyytyväiseyyden on todettu nousevan \cite{9533020}. Ketterää toteutetaan niin sanotuilla ketterillä menetelmillä, joista yleisimpien joukkoon lukeutuvat esimerkiksi Scrum \cite{SCRUMORG} ja XP \cite{XPORG}. Ketterää on haluttu kehittää entisestään tutkimalla ja kartoittamalla yleisimpiä haasteita ja niiden syitä \cite{7872736}.

Tutkimuksessaan Fitriani et al. \cite{7872736} totesivat yleisimmäksi haastealueeksi ketterää käyttävässä ryhmässä (jatkossa \textit{ketterä ryhmä}) hallinnalliset haasteet jaetulla ilmaantuvuudella hajautettuihin ryhmiin liittyvissä haasteissa, mitkä itsessään liittyvät vahvasti ensimmäiseen. Tutkimus julkaistiin vuoden 2016 ICACSIS-konferenssissa tavoitteenaan tuoda esille ketterään liittyviä yleisiä haasteita. Tutkimuksessa koottiin yhteen 20 eri haasteita käsittelevää tutkimusta, joista ilmeni kaiken kaikkiaan 30 haastetta. Yhdeksässä tutkimuksista käsiteltiin ketterään ohjelmistokehitysryhmään liittyviä hallinnallisia haasteita, joita ovat muun muassa suunnitteluun, koordinointiin, käytökseen sekä ryhmän käytänteisiin liittyvät huomiot. Ketterän ryhmän hallinnalliset haasteet ovat merkittäviä johtuen ketteristä periaatteista, joiden myötä muun muassa yksilö ja vuorovaikutus asetetaan tärkeämmäksi suhteutettuna prosesseihin ja työkaluihin \cite{beck2001agile}. Periaatteen myötä ketterä ryhmä on lähtökohtaisesti toiminnassaan autonominen, jonka myötä ryhmän hallinta ja suunnan ohjaus tapahtuu ryhmän toimesta. Hallinnallisuuteen liittyvät haasteet ovat myös merkittäviä siksi, että aihealueen on todettu olevan suurin vaikuttava tekijä tuottavuuteen \cite{DEOMELO2013412}.

Ketteryyteen liittyvään autonomisuuteen liittyy sekä vapautta muotoilla sopivat työskentelytavat että riskejä syöstä projekti pahimmassa tapauksessa raiteiltaan. Ilman ammattitaitoista ja ketteriin periaatteisiin tutustunutta ryhmää ryhmän projektin ohjaus ja ryhmänhallinta voivat olla hyvinkin haastavia \cite{7872736}. Vaikka ketterässä on monia hyviä puolia niin ryhmädynamiikan kuin asiakkaan kannalta, ketterän ryhmän toimintaan liittyy lukuisia mahdollisia haasteita. Haasteita ilmenee esimerkiksi tapauksissa, joissa ketterää yritetään käyttää tai toteuttaa puutteellisesti. Yksi merkittävimmistä haasteiden lähteistä on asetelma, jossa ketterää toteutetaan ymmärtämättä sen ydinperiaatteita. Toinen yleinen lähde haasteille on tilanne, jossa ketterää menetelmää käytetään puuttellisesti, minkä myötä joistain käytetyn menetelmän käytänteistä luovutaan tai jätetään vähemmälle. Esimerkkinä tilanteesta toimii niin sanottu ScrumBut \cite{SCRUMBUT}, jossa ketterä ryhmä on ottanut käyttöön Scrumin, mutta eivät toteuta sitä käytänteiltään täysimittaisesti esimerkiksi viikottaisten tapaamisten tai retrospektiivien suhteen.

Edellä pohditut skenaariot johtavat suuremmalla todennäköisyydellä haasteisiin ketterässä ryhmässä ja projektissa, minkä myötä projektin ja ketterän ryhmän hallinnasta voi tulla hyvinkin haastavaa. Haasteita käsiteltäessä on syytä huomioida ketterien ryhmien eri ryhmäkoot. Haasteet voivat esiintyä minkä kokoisessa ryhmässä tahansa niiden perustuessa samaan aatteeseen, mutta niiden voidaan olettaa muodostavan suurempaa riskiä ja potentiaalista vahinkoa suuremmissa variaatioissa, kuten maailmanlaajuisesti hajautetussa ketterässä kehityksessä \cite{ALZOUBI201622}. Tulevat kappaleet toimivat yleisimpien haastealueiden ja niihin liittyvien haasteiden esittelijöinä ja käsittelijöinä. Tietenkin on huomioimisen arvoista, että haasteet voivat mennä useamman kuin yhden kategorian alle, minkä myötä haastealueisiin jako ei ole täysin poissulkevaa. Tässä tutkielmassa keskitytään pohtimaan seuraavia kysymyksiä: \begin{enumerate}
    \item Mitä hallinnallisia haasteita liittyy ketterään ryhmään?
    \item Mitä ratkaisumalleja on olemassa ketterän ryhmän hallinnallisiin haasteisiin?
\end{enumerate}

\chapter{Koordinaatio}

\begin{figure}[t]
\centering 
\includegraphics[width=0.9\textwidth]{template/figures/koordinaatiohaasteet.pdf}
\caption{Kaavio koordinaatiohaasteista ja niiden tekijöistä perustuen Alzoubi et al. \cite{ALZOUBI201622}, de Melo et al. \cite{DEOMELO2013412}, Gregory et al. \cite{GREGORY201692}, Moe et al. \cite{MOE2012853} ja Silva et al. \cite{SELLERISILVA201520} tutkimuksiin.\label{fig:koordinaatiohaasteet}}
\end{figure}

Koordinaatiota voidaan tarkastella suurimpana vaikuttavana alueena ketterää ohjelmistokehitystä noudattavassa ryhmässä. Aihetta voidaan ajatella koko toimintaketjun sitovana liimana, sillä valtaosan esitettävistä aihealueista voidaan tulkita kuuluvan koordinaation alle. Tämä tutkielma jaottelee ryhmän koordinaatiohaasteet ketterän käyttöönottoon, kommunikaatioon, useamman ryhmän väliseen koordinaatioon sekä orientaatioon. Koordinaatio aiheena kattaa kaiken, joka voidaan tulkita vaikuttavan kehitysryhmän sisäiseen tai ulkoiseen yhteistoimintaan. Ryhmän sisäiseksi koordinaatioksi tulkitaan tässä kontekstissa ryhmäläisten välinen interaktio, kun taas ulkoiseksi koordinaatioksi tulkitaan esimerkiksi organisaation puolelta tuleva ohjeistaminen (tai ohjaus), mahdollisten muiden saman projektin parissa työskentelevien ryhmien kanssa yhteistyöskentely ja asiakkaan kanssa toimiminen. Esimerkkitapaus ulkoisesta koordinaatiosta on projekti, joka toteutetaan maailmanlaajuisesti hajautetulla ketteränä ohjelmistokehityksenä \cite{ALZOUBI201622}. Esitettävät haasteet voidaan tulkita vaikuttavan voimakkaammin niin laajemmissa ja maailmanlaajuisesti hajautetuissa ketterän kehityksen ryhmissä verrattuna niin sanotusti paikalliseen ryhmään. Maailmanlaajuisesti hajautettu ketterä kehitys tarkoittaa sitä, että prosessi koostuu ryhmistä tai ryhmäläisistä, jotka sijaitsevat maantieteellisesti kaukana toisistaan. Paikallinen ketterä kehitys tarkoittaa tässä kontekstissa vastakohtaa. Koordinaation ollessa varsin laaja käsite, siihen on yhdistettävissä lukuisia mahdollisia haasteita aina yhteistoiminnan käyttöönotosta ylläpitoon. Tulevissa alakappaleissa tutkielma esittelee kuvan \ref{fig:koordinaatiohaasteet} mukaisesti yleisimmät koordinaatiohaasteet, niiden tekijöitä ja vaikutuksia sekä mahdollisia ratkaisumalleja.

\section{Ketterän käyttöönotto}

Ensimmäisenä haasteena ketterän koordinointiin liittyen on ketterän menetelmän käyttöönotto ja siispä ketteriä periaatteita noudattavan ryhmän luonti. Eräänä haasteena käyttöönottoon liittyen on todettu tilanne, jossa perinteisiä malleja, kuten vesiputousta harjoittanut organisaatio haluaa ketteryyttä pohjautuen tietoon, että ketterä tuottaa parempaa kilpailukykyä \cite{MCKNIGHT2014168}. Haasteita tässä tapauksessa voi tuottaa organisaation ymmärrystaso ketteristä periaatteista ja niiden toteutustavoista. Puutteellinen ymmärrys voi olla seuraamusta käyttöönottoon liittyvästä puutteellisesta motivaatiosta, joka juontaa juurensa menetelmätottumuksiin \cite{GREGORY201692}. Syynä voi olla myös muutoksen vastustus tai liian haastelähtöinen näkökulma ketterään kehitykseen, kuten \cite{SELLERISILVA201520} toteavat. Mikäli organisaatiossa ei ymmärretä ketteryyteen liittyviä hyötyjä, siihen liittyviä riskejä tai varsinaisia toteutusmenetelmiä on odotettavissa haastava tie käyttöönoton suhteen.

Tutkimuksessaan Gregory et al. \cite{GREGORY201692} toteavat haasteita liittyen ketterän toteuttamiseen ja käyttöönottoon perinteisen ympäristön ja organisaation kontekstissa. Eräänä rajoittavana tekijänä käyttöönottoon liittyen todettiin ketterän väärinkäyttö, joka johtuu väärinymmärryksestä. Tutkimuksessa ilmeni tilanne väärinkäytöstä, missä organisaation johto on teettänyt niin sanotun ketterän ryhmän työskentelemään massiivisen määrän ylitöitä perustellen sen olevan osana ketterää menetelmää. Skenaariossa on nähtävissä kehittäjien loppuunpalaminen sekä organisaation paluu perinteisten menetelmien käyttöön, kun uusi menetelmä ei olekaan tuottanut odotettua tulosta. Toisena väärinkäyttöön liittyvänä skenaariona ilmeni tilanne, jossa jonkin ketterän menetelmän käyttöönoton jälkeen organisaation johto on alkanut mikromanageroimaan kehitysryhmää ainakin siihen saakka, kunnes nähtävää tulosta on alkanut muodostumaan. Jatkuvan mikromanageroinnin seurauksena on nähtävissä varsinaisen ketteryyden ja sen sisältävän luovuuden väheneminen. Toisaalta ohjaukseen liittyen Silva et al. \cite{SELLERISILVA201520} toteavat olevan oleellista tukea ja ohjata ketterän käyttöönottoa aikaisissa vaiheissa.

Käyttöönottohaasteiden minimoimiseksi on todettu olevan suositeltavaa hakea konsultointia ammattilaisilta ja yrityksiltä, joilla on tunnustettua erityisosaamista ketterään kehitykseen liittyen \cite{SELLERISILVA201520}. Siispä ratkaisuna haasteelle voidaan ajatella olevan tehokasta perehtyä kattavasti ketteriin periaatteisiin ja käyttöönotettavaan menetelmään. Silva et al. \cite{SELLERISILVA201520} muistuttavat lisäksi, että ketterien periaatteiden mukaisesti ketterän kehitysryhmän tulisi pyrkiä itse esittämään ratkaisumalleja ongelmatilanteisiin.

\section{Kommunikaatio}

Kommunikaation voidaan ajatella olevan ketterän ryhmän koordinaation mahdollistava työkalu. Ilman kommunikaatiota ei olisi yhteistoimintaa, ainakaan tehokkaanlaatuista. Alzoubi et al. \cite{ALZOUBI201622} korostavat kommunikaation tehokasta käyttöönottoa alusta alkaen muiden haasteiden välttämiseksi. Haasteiden minimoimiseksi olisi olennaista, että ketterä ryhmä kommunikoi ryhmäläistensä kesken ja ryhmän ulkopuolella olevien yhteistyötahojen kanssa. Kommunikaation merkitys korostuu maantieteellisesti hajautettujen ryhmien osalta, sillä kyseiset kehittäjät eivät lähtökohtaisesti pääse samaan työskentelytilaan muiden kanssa. Ketterien periaatteiden korostaessa asiakasyhteistyötä on myös ratkaisevaa, että ryhmä kommunikoi asiakkaan kanssa riittävän selkeästi ja usein.

Tutkimuksessaan Alzoubi et al. \cite{ALZOUBI201622} keskittyvät ketterään liittyvän kommunikaatioon ja tuovat esille siihen liittyviä haasteita ja ratkaisumalleja. Tutkimus itsessään keskittyy maailmanlaajuisesti hajautetun ketterän ohjelmistokehityksen kommunikatiivisiin seikkoihin, mutta tuloksia voidaan soveltaa paikallisesti toteutettavaan variaatioon oletuksella, että paikallisvariaatiossa haasteet ovat lievempiä. Tutkimuksessa puutteellisen kommunikoinnin todettiin johtavan kaikkiin muihin haasteisiin ja niiden käsittelyn vaativan 2,5 kertaa enemmän resursseja kuin paikallisessa variaatiossa. Alzoubi et al. toteavat ketterän kehityksen nojaavan epävirallisiin kommunikointikeinoihin ryhmäläisten välillä, sillä kommunikoinnin on todettu olevan huomattavasti tehokkaampaa epävirallisten menetelmien pohjalta. Epäviralliset kommunikointimenetelmät voidaan jakaa henkilökohtaiseen, vuorovaikutteiseen ja vertaisorientoituneeseen tapaan. Epävirallisen kommunikoinnin on todettu nojaavan voimakkaasti vuorovaikutukseen, joka tapahtuu kasvotusten niin ryhmäläisten kuin ryhmän ja asiakkaan välillä. Tästä johtuen ryhmäläisten välinen sijainti korostuu voimakkaasti vaikuttavana tekijänä.

Alzoubi et al. tutkimuksessa \cite{ALZOUBI201622} todettiin, että lähteinä kommunikointihaasteille ovat muun muassa etäisyys, organisaatiotekijät, ihmistekijät ja ryhmäkonfigurointi. Etäisyyden todettiin olevan yleisin haasteiden lähde. Sen todettiin hankaloittavan koordinaatiota esimerkiksi viiveinä kommunikoinnissa, pidempinä kokouksina, hankaluutena löytää yhteisiä työaikoja ja luottamuksen puutteena. Silva et al. tutkimuksessa \cite{SELLERISILVA201520} todettiin rajoittavana tekijänä ryhmän sisäiset erimielisyydet, joiden voidaan olettaa hankaloittavan kommunikointia. Ryhmäkonfigurointiin liittyen sekä \cite{SELLERISILVA201520} että \cite{ALZOUBI201622} toteavat kommunikoinnin hankaloituvan sen mukaan, mitä suurempi ryhmä on kyseessä. Sama pätee myös tapaukseen, jossa projektin parissa työskentelee useampi ryhmä. Lisäksi Alzoubi et al. \cite{ALZOUBI201622} toteavat, että satunnaisissa tapauksissa jotkin ryhmäläiset eivät halua kommunikoida ja omaavat huonomman ymmärryksen koordinaatiosta. Kommunikoinnin haluttomuuteen liittyy myös Moe et al. \cite{MOE2012853} tutkimus, jossa todetaan haasteena ryhmäläisten keskinäinen kohtaaminen. Ryhmäläiset eivät välttämättä halua väitellä keskenään, vaan mieluummin mukautuvat tilanteeseen. Seurauksena haasteista ryhmäkonfiguraatiossa on muun muassa aikaisen vaiheen kommunikointiongelmat, ryhmäläisten haluttomuus ryhmäkommunikointiin, vähentynyt ymmärrys ryhmätyöskentelystä sekä hitaus kommunikoinnissa \cite{ALZOUBI201622}. Lisäksi keskinäisen kohtaamattomuuden seurauksena on ryhmän tehoton päätöksenteko \cite{MOE2012853}. Organisaatiotekijöiksi listattiin organisaation työkalut ja palveluiden kyky tukea kommunikointia \cite{ALZOUBI201622}. Työkaluihin ja palveluihin lukeutuvat muun muassa puhelin, pikaviestipalvelut sekä sähköposti. Merkittäväksi tekijäksi on listattu myös organisaation kulttuuri, joka koostuu asenteista, arvoista ja käytänteistä. Seurauksena organisaatiotekijöistä ovat muun muassa tehottomampi ryhmä, huonolaatuisempi ohjelmisto, suurempi määrä työkaluja sekä puutteellinen luotto.

Ratkaisumalleiksi Alzoubi et al. \cite{ALZOUBI201622} listaavat seuraavanlaista. Ryhmäkonfiguraatiosta lähtöisin oleviin haasteisiin strategiaksi ehdotetaan ensinnäkin paikallisten tapaamisten järjestämistä. Paikallisten tapaamisten on todettu tuottavan epävirallisempaa kommunikointia, joka tehostaa ryhmän tuottavuutta \cite{DEOMELO2013412}. Toisaalta \cite{ALZOUBI201622} listaavat samalla osaksi strategiaa tehokkaan virallisen kommunikoinnin lisäämisen. Muita osia strategiasta ovat muun muassa projektin säännöllinen esittely osapuolille, säännöllisten tapaamisten järjestäminen, paikallisten ryhmien muodostus sekä selkeiden roolien ja vastuiden jako. Toisaalta \cite{CLAPS201521} toteavat roolien jakoon liittyen erittäin oleelliseksi oikeanlaisen ohjauksen mahdollisen johdon puolelta, ettei prosessi ole vahingollinen ryhmille. Claps et al. \cite{CLAPS201521} korostavat roolijaon olevan itsessään haastava tehtävä. Organisaatiotekijöistä lähtöisin oleviin haasteisiin Alzoubi et al. \cite{ALZOUBI201622} ehdottavat strategiaksi korostaa työskentelykulttuuria, joka suosii jatkuvaa kommunikointia, luottoa sidosryhmien välillä sekä nopeaa asiakaspalautetta. 

\section{Useamman ryhmän yhteistoiminta}

Tutkimuksessaan de Melo et al. \cite{DEOMELO2013412} toteavat ketterän ryhmän tuottavuuden koostuvan ryhmän sisäisistä ja ulkoisista tekijöistä. Ryhmän ulkoiset tekijät rajautuvat useamman ryhmän väliseen koordinaatioon. On todettu, että useista kehitysryhmistä koostuvien projektien ryhmien välinen koordinaatio on yksi hankalimmista kehitettävistä asioista ohjelmistotuotantoon liittyen. Ryhmien välisen koordinoinnin haasteita aiheuttavat sitoutumisen puute sekä epäsopivat koordinointisäännöt ryhmien välillä. Haasteiden on todettu aiheuttavan suuntausvirheitä ja ketteryyden rikkoutumisen. Esimerkkinä suuntausvirheestä toimii tilanne, jossa ryhmän työskentely riippuu toisen ryhmän tuloksista, eivätkä ryhmät työskentele samassa tahdissa. Tällöin syntyy tilanteita, joissa toisesta ryhmästä riippuva ryhmä joutuu odottamaan tai hidastamaan tahtiaan merkittävästi. Toisesta ryhmästä riippuvuuden on todettu olevan yleinen ilmiö laajemmissa ohjelmistoprojekteissa.

Laajojen, useista kehittäjäryhmistä koostuvien projektien osalta on todettu olevan haastavaa toteuttaa laaja-alaista ketterää kehitystä \cite{DEOMELO2013412}. Kaikki ryhmät eivät välttämättä noudata ketteriä periaatteita tai käytä mitään ketterää menetelmää. De Melo et al. \cite{DEOMELO2013412} toteavat, että ketterän ryhmän on äärimmäisen hankalaa tai jopa mahdotonta vaikuttaa muiden ryhmien työskentelymenetelmiin. Täten tilanne, jossa osa ryhmistä käyttää eri ketteriä menetelmiä, osa vaikkapa vesiputousmallia, on mahdollinen. Projektin useat eri kehitysmenetelmät voivat aiheuttaa massiivisia koordinointiongelmia. De Melo et al. mainitsevat ratkaisuna useamman ryhmän yhteistoimintaan ketterän ryhmän kyvyn mukautua erilaisiin tilanteisiin. Mukautumisen kautta voisi olla mahdollista löytää paremmin yhteisiä työskentelytapoja eri menetelmiä käyttävien ryhmien kanssa erityisesti tilanteessa, jossa kumpikin (tai useampi) osapuoli noudattaa jotain ketterää menetelmää. Claps et al. \cite{CLAPS201521} mainitsevat, ettei ryhmien välinen koordinaatio olisi toiminut uuden ketterän menetelmän käyttöönoton kontekstissa ilman johdon käyttöönottamaa strategiaa. Organisaation johdon selkeän strategian myötä kullekin ryhmälle saatiin yhteinen tavoite, jonka myötä ryhmät työskentelivät samaa päämäärää kohti.

\section{Orientaatio}

Tutkimuksessaan Moe et al. \cite{MOE2012853} toteavat yhtenä koordinaatiohaasteena olevan ryhmäorientaation puute. Ryhmäorientaatio tarkoittaa ryhmän yhteistä suuntaa ja sen toteuttaminen tarkoittaa sitä, että kukin ryhmässä työskentelee yhteistä päämäärää kohti. Tutkimuksen mukaan joissain tilanteissa ketterän ryhmän orientaatio voi olla alhainen epärealististen tavoitteiden takia. Alhaisen ryhmäorientaation myötä on todettu tapauksia, joissa yksittäiset ryhmäläiset tekevät omia päätöksiä projektin suhteen sen sijaan, että päätöksistä sovittaisiin yhdessä. Samaten ongelmien raportoinnin on todettu jäävän vähemmälle johtuen siitä, että alhaisen orientaation myötä projektiin liittyviä ongelmia pidetään enemmän henkilökohtaisina kuin ryhmän ongelmina. Kaiken kaikkiaan alhainen ryhmäorientaatio vähentää ryhmän kommunikointia merkittävästi ja sen voidaan tulkita lisäävän riskiä ryhmäläisten siiloutumiselle. Siiloutumisella tarkoitetaan sitä, että jokin osa ryhmästä alkaa työskentelemään suljetusti keskenään ilman muuta osaa ryhmästä.

\chapter{Suunnittelu}

Suunnittelun merkitys nykyaikaisessa ohjelmistokehityksessä on vähentynyt perinteisiin malleihin, kuten vesiputousmalliin nähden. Toisaalta puutteet suunnittelussa on kuitenkin yksi yleisimmistä ketterää toteuttavan ryhmän hallinnallisten haasteiden lähteistä \cite{7872736}, minkä myötä sitä ei kuitenkaan saa unohtaa. Beck et al. \cite{beck2001agile} määrittävät yhtenä ketteristä periaatteista, että muutokseen reagointi on tärkeämpää kuin suunnitelmassa pysyminen. Suunnitteluun liittyvät haasteet voidaan jakaa karkeasti ryhmäkokoonpanon suunnitteluun, joka itsessään jakautuu de Melo et al. \cite{DEOMELO2013412} mukaisesti ryhmän sisäisiin ja ulkoisiin tekijöihin.

\section{Ryhmäkokoonpano}

Tutkimuksessaan de Melo et al. \cite{DEOMELO2013412} toteavat ketterän ryhmän hallinnan olevan suurin vaikuttava tekijä tuottavuuteen. Tutkimus kartoitti merkittävimmät tuottavuuteen vaikuttavat tekijät, minkä myötä ne rajattiin ketterän ryhmän sisäisiin ja ulkoisiin haasteisiin. Ryhmän sisäisistä haasteista merkittävimmiksi korostuivat ryhmän suunnitteluvalinnat sekä jäsenten vaihtuvuus, kun taas ulkoisista haasteista merkittävimmäksi muodostui ryhmien välinen koordinaatio.

[Katso löytyisikö seuraavaan kohtaan jotain lähteitä tukemaan vielä olettamusväitteitä, jotta olisi suorempaa viitettä]

\subsection{Suunnitteluvalinnat}

Ryhmän suunnitteluvalinnat koostuvat \cite{DEOMELO2013412} tutkimuksessa ryhmäkoosta, ryhmäläisten taidoista, ryhmäläisten keskinäisestä sijainnista sekä ryhmäläisten jaosta. Tutkimuksesta on tulkittavissa, että haasteita kokoonpanoon tuottaa tässä kontekstissa muun muassa seuraavanlaiset tilanteet. Kokoonpanossa, jossa valtaosa ryhmäläisistä olisi osa-aikaisia työntekijöitä keskinäinen keskittyminen projektiin voisi olla heikkoa perustuen tutkimuksen toteamukseen tapauksesta, jossa täyspäiväjäsenillä on todetusti parempi keskittyminen. [Lisää järkevästi muotoiltuja kohtia tähän liittyen]

\subsection{Vaihtuvuus}

Ryhmän jäsenten vaihtuvuus on toinen ryhmän sisäistä haasteista \cite{DEOMELO2013412}. Vaihtuvuudeksi de Melo et al. määrittävät sekä aiemman ryhmäläisen lähtönä ryhmästä että uuden kehittäjän liittymisenä ryhmään. Vaihtuvuus aiheuttaa organisaatiolle monenlaisia kustannuksia, kuten irtisanoutumiseen liittyvät, rekrytointiin sekä perehdyttämiseen liittyviä menoja. Lisäksi kehitysryhmä työskentelee vaihtuvuuden aikana astetta pienemmällä teholla. Tutkimuksessaan de Melo et al. määrittävät vaihtuvuuden tekijöiksi palkkatason sekä kehittäjän sopeutumattomuutena ryhmään. Sopeutumattomuuden on arvioitu johtuvan muun muassa oma-aloitteisuuden ja sitoutumisen puutteesta, epäsopivasta personaallisuudesta sekä ryhmän sisäisistä konflikteista. Viimeisestä on tulkittavissa, että haasteet kommunikoinnissa voivat aiheuttaa suunnitelmallisia haasteita.

Vaikka vaihtuvuuden on todettu aiheuttavan suunnitelmallisia haasteita, niin sen on todettu myös vaikuttavan positiviisesti uusien ryhmäläisten kautta \cite{DEOMELO2013412}. On todettu, että uudet ryhmäläiset saattavat tuoda uusia ideoita, ratkaisuja ja yleisesti ottaen energiaa ryhmään. Toisaalta vaihtuvuuden negatiivisia vaikutuksia on lukumäärällisesti enemmän ja ne voidaan jakaa pidentyneeksi tuotantoajaksi lyhyellä aikavälillä, oleellisen tiedon menettämisenä poistumisen yhteydessä sekä lisäkustannuksina. 

\chapter{Yksilömuuttujat}

Viimeisenä rajauksena aihealueisiin toimii yksilömuuttujat. Nimensä mukaisesti yksilömuuttujissa on kyse ketterän ryhmän hallinnallisista haasteista, jotka aiheutuvat yksilöstä. Yleisimmiksi yksilöhaasteiksi tutkimuksista [lähteet] lukeutuivat yksilöllinen motivaatio sekä käytökselliset seikat.
\chapter{Yhteenveto\label{conclusions}}

[1-2 sivua]

[Todetaan johdannon merkityksellisyys kokonaisaiheessa]

[Mainitaan tutkimuskysymykset ja niihin ilmenneet vastaukset.]

[Toistetaan kunkin haastealueen keskeisimmät haasteet ja niiden mahdolliset ratkaisumallit]

[Pohdintaa esimerkiksi liittyen tarpeesta jatkotutkimuksille esim. ratkaisumallien osalta.]

[Maininta alkper lähteestä: Tiimihallinta on erittäin haastavaa ja vaatii erityishuomiota softaprojektin onnistumisen takaamiseksi]

%%%%%%%%%%%%%%%%%%%%%%%%%%%%%%%%%%%%%%%%%%%%%%%%%%%%%%%%%
%\cleardoublepage                         % Fixes the position of bibliography in bookmarks
%\phantomsection
\addcontentsline{toc}{chapter}{\bibname}  % This lines adds the bibliography to the ToC
\printbibliography

%%%%%%%%%%%%%%%%%%%%%%%%%%%%%%%%%%%%%%%%%%%%%%%%%%%%%%%%%
\backmatter
\begin{appendices}

%% A sample Appendix
%% \include{Ch.90_Appendix_1}

% BSc instructions
%\include{instructions/bsc_finnish_contents}

\end{appendices}
%%%%%%%%%%%%%%%%%%%%%%%%%%%%%%%%%%%%%%%%%%%%%%%%%%%%%%%%%

\end{document}
