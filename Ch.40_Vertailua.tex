\chapter{Tutkielman, haasteiden ja ratkaisumallien arviointia}

Tässä kappaleessa tutkielma keskittyy arvioimaan omaa validiteettiaan ja reliabiliteettiaan sekä vertailemaan esitettyjä haastealueita ja niihin liittyviä mahdollisesti esitettyjä ratkaisumalleja. Tutkielman validiteetilla tarkoitetaan sitä, että arvioidaan kykeneekö tutkielma vastaamaan asetettuihin tutkimuskysymyksiin. Reliabiliteetilla tarkoitetaan sitä, että arvioidaan onko tutkielma luotettava ja tarkasteleeko se asioita esimerkiksi tarpeeksi objektiivisesti.

\section{Yleisesti}

Tutkielma jakoi haastealueet koordinaatioon ja kokoonpanoon pohjautuen Fitriani et al. \cite{7872736} tutkimukseen. Fitriani et al. tutkimus määritti hallinnallisiin haasteisiin kuuluvan kaikki ryhmän suunnitteluvalintoihin, koordinaatioon, käytökseen sekä ryhmän muihin käytänteisiin liittyvät haasteet. Tässä tutkielmassa haasteet rajattiin kahteen aihealueeseen: koordinaatioon ja kokoonpanoon. Kolmantena aihealueena olisi voinut olla yksilömuuttujiin pureutuva kappale, sillä useammassa tutkimuksessa ilmeni yksilön haasteiden vaikutus ryhmän hallintaan. Yksilömuuttujiin pureutuva kappale jätettiin pois sen keskittyessä enemmän yksilöön ja psykologiaan enemmän kuin varsinaiseen ryhmään. Tässä tutkielmassa ilmenevästä haastealueiden jaosta kahteen aiheeseen on syytä huomioida, että jako ei ole poissulkeva. Koordinaation tarkoittaessa kaikkea yhteistoimintaa mahdollistavaa, aiheen alle voi tulkita näkökulmasta riippuen melkein mitä vain. Voitaisiin esimerkiksi väittää, että tutkielman kolmas kappale, joka pureutuu kokoonpanohaasteisiin, lukeutuisi koordinaatioon, sillä ryhmäkokoonpano vaikuttaa huomattavasti ryhmän yhteistoimintaan. Edeltävänkaltaiset argumentit tiedostaen tutkielma ilmaisee jaon olevan karkea, eikä poissulje haasteiden kuuluvuutta useampaan kuin yhteen alueeseen. Lisäksi on huomioitavaa, että tämä tutkielma listaa vain yleisimmät haasteet rajallisesta määrästä lähteitä huomioiden, että kyseessä on varsin lyhyt kanditutkielma. Vielä lisäksi on huomioimisen arvoista, että haasteiden määrittelyyn käytetyt lähteet ovat jo usean vuoden vanhoja niiden ollessa keskimäärin vuosilta 2015 ja 2016. Tämän pohjalta tässä tutkielmassa ei välttämättä käsitellä sellaista oleelliseksi tulkittavaa tietoa niin haasteisiin kuin ratkaisuihinkaan liittyen, mikäli niihin liittyvä tieto on julkaistu lähivuosina.

\section{Haastealueet}

Tutkielman haastealueet jakautuvat ryhmän koordinaatioon ja kokoonpanoon. Koordinaatioon liittyvä kappale käsittelee yleisimmiksi ilmenneitä ryhmän (tai ryhmien) yhteistoimintaan vaikuttavia haasteita, kun taas kokoonpanoon liittyvä kappale käsittelee yleisimpiä ryhmäkokoonpanoon liittyviä haasteita. 

Kuvan \ref{fig:koordinaatiohaasteet} mukaisesti koordinaatiohaasteiksi määriteltiin ketterän käyttöönotto, ryhmän kommunikointi, useamman ryhmän välinen koordinaatio sekä ryhmäsuuntautumisen puute. Käyttöönottohaasteeseen liittyy tekijöinä ryhmän sisällä ilmenevä muutoksen vastustus, ketterien periaatteiden tai menetelmien väärinymmärrys sekä väärinkäyttö. Ketterien periaatteiden väärinymmärryksen todettiin johtavan väärinkäyttöön. Kommunikaation yhteyteen liittyy tekijöinä ryhmän erimielisyydet, yksilöllisen kohtaamisen puute, ihmistekijät, ryhmäläisten keskinäinen etäisyys, organisaatiotekijät sekä ryhmäkonfigurointi. Lisäksi erimielisyyksien todettiin johtavan yksilöllisen kohtaamisen haasteisiin. Useamman ryhmän väliseen koordinaatioon liittyen tekijöiksi todettiin ryhmäläisten (tai ryhmien) sitoutumisen puute sekä epäsopivat koordinointisäännöt. Lopulta orientaatiohaasteen ainoaksi tekijäksi todettiin ryhmän epärealistiset tavoitteet.

Kuvan \ref{fig:kokoonpanohaasteet} mukaisesti kokoonpanohaasteiksi lueteltiin ryhmän suunnitteluvalinnat sekä ryhmäläisten vaihtuvuus. Suunnitteluvalintaan todettiin vaikuttavan ryhmän koko, ryhmäläisten taidot, ryhmäläisten keskinäinen sijainti sekä ryhmäjako. Vaihtuvuuden tekijöiksi määriteltiin palkkataso sekä ryhmäläisen sopeutumattomuus. 

Kaiken kaikkiaan koordinaatiohaasteita ilmeni huomattavasti enemmän kuin kokoonpanohaasteista, vaikkakin teoriassa kokoonpanohaasteet voitaisiin tulkita koordinaatiohaasteiksi. Kummankin haastealueen yhteydessä organisaatiokonteksti tuntui korostuvan useampaan otteeseen, johon Silva et al. \cite{SELLERISILVA201520} tutkimuksessa todettiinkin, että organisaatiokontekstissa hallinnan ylläpito on haastavaa ketteryyttä tukien.

\section{Ratkaisumallit}

Haastealueisiin liittyvien ratkaisumallien lukumäärä oli tutkimuksissa varsin vähäinen. Osa viitatuista tutkimuksista listasi haasteita enimmäkseen ongelmalähtöisestä näkökulmasta ratkaisulähtöisen mallin sijaan. Lisäksi osa haasteista ja esitetyistä malleista ovat hieman ristiriitaisia tai tulkinnanvaraisia. Esimerkiksi tutkimuksessaan käyttöönoton haasteisiin liittyen Gregory et al. \cite{GREGORY201692} haastavaksi organisaatiojohdon mikromanageroinnin, kun taas Silva et al. \cite{SELLERISILVA201520} korostavat ohjauksen olevan tärkeää käyttöönoton aikaisissa vaiheissa. Tämän myötä vastuu ohjauksen määrästä jää sekä ohjaajan tilannetajun ja ketterien periaatteiden ymmärryksen sekä ryhmän kommunikoinnin varaan. Toisaalta yleisesti ottaen käyttöönottohaasteisiin Silva et al. suosittelevat ratkaisuksi konsultoinnin hakemista ammattilaisilta, mutta korostavat myös ongelmanratkaisun olevan tärkeää ryhmän aloitteesta. Kommunikaatiohaasteisiin liittyen Alzoubi et al. \cite{ALZOUBI201622} ehdottavat ratkaisuksi lähitapaamisten järjestämistä korostaen epävirallista tapaa kommunikoida. Toisaalta ristiriitaisesti Alzoubi et al. ehdottavat osana ratkaisumallia virallisen kommunikointitavan määrän lisäämistä. Lisäski Alzoubi et al. listaavat kommunikaatiohaasteiden ratkaisuiksi demotilaisuuksien lisäämisen, paikallisryhmien muodostuksen sekä selkeän roolijaon. Toisaalta Claps et al. \cite{CLAPS201521} ovat todenneet roolijaon olevan haasteellista, jonka myötä se saattaisi aiheuttaa lisähaasteita. Lopulta Alzoubi et al. \cite{ALZOUBI201622} ehdottavat organisaatikontekstin kommunikointihaasteiden ratkaisuksi jatkuvan kommunikoinnin suosivaa työkulttuuria, luottamuksen lisäämistä sekä lisäkommunikointia asiakkaan kanssa. Useamman ryhmän koordinaation haasteisiin de Melo et al. \cite{DEOMELO2013412} toteavat tehokkainta olevan ketterän ryhmän kyvyn mukautua tilanteeseen kun taas Claps et al. \cite{CLAPS201521} korostavat organisaatiojohdon ohjeistuksen merkitystä yhteisen päämäärän saavuttamiseksi. Orientaatiohaasteelle ei tässä tutkielmassa löytynyt ratkaisumallia, mutta todennäköisimmän ratkaisun voi päätellä liittyvän ryhmän avoimempaan kommunikointiin.

