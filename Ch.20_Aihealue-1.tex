\chapter{Haasteet\label{aihe-1}}

Tässä kappaleessa käsitellään ketterään ryhmään liittyviä hallinnallisia haasteita. Vaikka ketterän on todettu parantavan muun muassa kehitysprosessia, lopputulosta sekä asiakastyytyväisyyttä, ketterän ryhmän toimintaan liittyy lukuisia mahdollisia haasteita \cite{7872736}. Haasteita ilmenee esimerkiksi tapauksissa, joissa ketterää yritetään käyttää tai toteuttaa puutteellisesti. Yksi merkittävimmistä haasteiden lähteistä on asetelma, jossa ketterää toteutetaan ymmärtämättä sen ydinperiaatteita. Toinen yleinen lähde haasteille on tilanne, jossa ketterää menetelmää käytetään puuttellisesti, minkä myötä joistain käytetyn menetelmän käytänteistä luovutaan tai jätetään vähemmälle. Esimerkkinä tilanteesta toimii niin sanottu ScrumBut \cite{SCRUMBUT}, jossa ketterä ryhmä on ottanut käyttöön Scrumin, mutta eivät toteuta sitä käytänteiltään täysimittaisesti esimerkiksi viikottaisten tapaamisten tai retrospektiivien suhteen.

Edellä pohditut skenaariot johtavat suuremmalla todennäköisyydellä haasteisiin ketterässä ryhmässä ja projektissa, minkä myötä projektin ja ketterän ryhmän hallinnasta voi tulla hyvinkin haastavaa. Haasteita käsiteltäessä on syytä huomioida ketterien ryhmien eri ryhmäkoot. Haasteet voivat esiintyä minkä kokoisessa ryhmässä tahansa niiden perustuessa aatteeseen, mutta niiden voidaan olettaa muodostavan suurempaa riskiä ja potentiaalista vahinkoa suuremmissa variaatioissa, kuten maailmanlaajuisesti hajautetussa ketterässä kehityksessä \cite{ALZOUBI201622}. Tulevat alakappaleet toimivat yleisimpien haastealueiden ja niihin liittyvien haasteiden esittelijöinä ja käsittelijöinä. Tietenkin on huomioimisen arvoista, että haasteet voivat mennä useamman kuin yhden kategorian alle, minkä myötä haastealueisiin jako ei ole täysin poissulkevaa. 


\section{Käyttöönotto}

Ketterän käyttöönottoon niin pienellä kuin isommallakin ryhmällä liittyy lukuisia haasteita. Juurihaasteena käyttöönotossa toimii puutteellisuus ketterien menetelmien ja periaatteiden ymmärrykseen, joka voi olla seuraamusta motivaation puutteesta käyttöönottoon \cite{GREGORY201692}. Tähän liityyen on pääteltävissä tilanne, jossa organisaatio tiedostaa ketterän tuottavan parempaa kilpailukykyä, mutta on harjoittanut vesiputousmallia \cite{MCKNIGHT2014168} pidemmän aikaa. Organisaatiossa pidemmän aikaa vesiputousaatteella työskennellyt saattaa olla haluton muutokseen, joka itsessään on yksi hallinnallisista haasteista. Sivussa motivaatiokysymyksestä mikäli organisaatio tai kehitysryhmät eivät ymmärrä ketterän hyötyjä, siihen liittyviä riskejä tai varsinaisia toteutusmenetelmiä on odotettavissa kivinen käyttöönotto, joka ei välttämättä mene maaliin asti.

Tutkimuksessaan Gregory et al. \cite{GREGORY201692} toteavat useita haasteita liittyen ketterän toteuttamiseen ja käyttöönottoon epä-ketterän ympäristön ja organisaation kontekstissa. Eräänä rajoittavana tekijänä käyttöönottoon liittyen todettiin ketterän väärinkäyttö tai -ymmärrys. Esimerkkinä väärinkäytöstä toimii tutkimuksessa esiin tullut tilanne, jossa organisaation johto on pakottanut ketterän ryhmän työskentelemään massiivisen määrän ylitöitä perustellen sen olevan osaa ketterää aiheuttaen kehittäjille työuupumusta. Toisena esimerkkinä väärinkäytöstä ilmeni tilanne, jossa ketterän käyttöönoton jälkeen johto on mikromanageerannut ketterää ryhmää ainakin siihen saakka, kunnes nähtävää tulosta on alkanut muodostumaan. Jatkuvan mikromanageroinnin seurauksena on nähtävissä varsinaisen ketteryyden ja sen sisältävän luovuuden väheneminen.


\section{Kommunikaatio}

Kommunikaatio, jossa voidaan käsitellä kulttuurieroja

Ketterän ryhmän hallinnallisten haasteiden keskiössä on kommunikaatioon liittyvät ongelmat, jotka korostuvat erityisesti maailmanlaajuisella skaalalla toteutetussa ryhmävariaatiossa. Kommunikaatioon liittyvät haasteet voidaan jakaa ryhmän sisäiseen kommunikaatioon sekä ryhmän ja asiakkaan välillä tapahtuvaan kommunikaatioon. 


\section{Suunnittelu / koordinaatio}

Kolmantena haastealueena ketterien ryhmien hallinnassa on koordinaatio, joka kattaa muun muassa ryhmän ja muiden osapuolten välisen yhteistoiminnan, ryhmäkokoonpanon suunnittelun sekä tehtävien jaon.

Tutkimuksessaan de Melo et al. \cite{DEOMELO2013412} toteavat 
