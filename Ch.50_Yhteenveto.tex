\chapter{Yhteenveto\label{conclusions}}

Tutkielmassa ketterän ohjelmistokehitysryhmän hallinnalliset haasteet todettiin merkittäviksi neljästä syystä. Ensinnäkin ketterän ryhmän hallinnalliset haasteet ovat merkittäviä ketterien periaatteiden takia \cite{beck2001agile}. Toiseksi ketterän ryhmän hallinta on yleisesti käsitelty aihe ketterään kehitykseen liittyvissä haasteissa \cite{7872736}. Kolmanneksi ryhmänhallinnan on todettu olevan suurin vaikuttava tekijä ketterän ryhmän tuottavuuteen \cite{DEOMELO2013412}. Viimeisenä ryhmänhallinnan on arvioitu olevan erittäin haastavaa ja vaativan erityishuomiota ohjelmistoprojektin onnistumisen takaamiseksi \cite{7872736}. Tutkielma pohti kahta tutkimuskysymystä: mitä hallinnallisia haasteita liittyy ketterään ryhmään ja mitä ratkaisumalleja on olemassa ketterän ryhmän hallinnallisiin haasteisiin? Hallinnallisten haasteiden todettiin jakautuvan koordinaatioon ja kokoonpanoon kuvien \ref{fig:koordinaatiohaasteet} ja \ref{fig:kokoonpanohaasteet} mukaisesti. Merkittävimmiksi koordinaatiohaasteiksi muodostuivat ketterän käyttöönotto, ryhmäkommunikaatio, useamman ryhmän välinen yhteistoiminta sekä ryhmäorientaatio. Merkittävimmiksi kokoonpanohaasteiksi lukeutuivat ryhmän suunnitteluvalinnat sekä ryhmäläisten vaihtuvuus. Keskeisimmiksi ratkaisumalleiksi esitettiin konsultoinnin hakemista ammattilaisilta, ryhmälähtöistä ongelmanratkaisua, epävirallisten kommunikointimenetelmien ja lähitapaamisten suosimisen, virallisen kommunikointitavan lisäämistä, selkeän roolijaon ja paikallisryhmien muodostuksen, jatkuvan kommunikoinnin työkulttuurin suosimisen, luottamuksen lisäämisen, lisäkommunikoinnin asiakkaan kanssa, ryhmän kyvyn mukautua tilanteisiin ja organisaatiojohdon sopivan tason ohjeistusta. Lopuksi vertailukappaleessa jatkotutkimukset todettiin tarpeelliseksi ratkaisumalleihin keskittyen.
