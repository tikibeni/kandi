\chapter{Yhteenveto\label{conclusions}}

Tutkielmassa ketterän ohjelmistokehitysryhmän hallinnalliset haasteet todettiin merkittäviksi johtuen niiden yleisestä ilmenemisestä ketterää ohjelmistokehitystä koskevien haasteiden käsittelevissä tutkimuksissa \cite{7872736}, ryhmänhallinnan olevan suurin vaikuttava tekijä ketterän ryhmän tuottavuuteen \cite{DEOMELO2013412} sekä sen takia, että ryhmänhallinnan on todettu olevan erittäin haastavaa ja vaativan erityishuomiota ohjelmistoprojektin onnistumisen takaamiseksi \cite{7872736}. Tutkielma pohti kahta tutkimuskysymystä, jotka olivat: "Mitä hallinnallisia haasteita liittyy ketterään ryhmään?" sekä "Mitä ratkaisumalleja on olemassa ketterän ryhmän hallinnallisiin haasteisiin?". Hallinnallisten haasteiden todettiin jakautuvan koordinatiivisiin ja kokoonpanollisiin alueisiin kuvien \ref{fig:koordinaatiohaasteet} ja \ref{fig:kokoonpanohaasteet} mukaisesti. Merkittävimmiksi koordinaatiohaasteiksi muodostuivat ketterän käyttöönotto, ryhmäkommunikaatio, useamman ryhmän välinen yhteistoiminta sekä ryhmäorientaatio. Merkittävimmiksi kokoonpanohaasteiksi lukeutuivat ryhmän suunnitteluvalinnat sekä ryhmäläisten vaihtuvuus. Keskeisimmiksi ratkaisumalleiksi esitettiin konsultoinnin hakemista ammattilaisilta, ryhmälähtöistä ongelmanratkaisua, epävirallisten kommunikointimenetelmien ja lähitapaamisten suosimisen, virallisen kommunikointitavan lisäämistä, selkeän roolijaon ja paikallisryhmien muodostuksen, jatkuvan kommunikoinnin työkulttuurin suosimisen, luottamuksen lisäämistä, lisäkommunikoinnin asiakkaan kanssa, ryhmän kyvyn mukautua tilanteisiin ja organisaatiojohdon sopivan tason ohjeistusta. Lopuksi vertailun ohessa jatkotutkimukset todettiin tarpeelliseksi ratkaisumalleihin keskittyen.
