\begin{abstract}

Ketterä ohjelmistokehitys on kehitetty korjaamaan perinteisten tuotantomenetelmien ja -mallien ongelmakohtia. Vaikka ketterä manifesti korjasi ketterää edeltävän iteratiivisen mallin aatteelliset ongelmat, ketterä kehitys sisältää omat haastealueensa, joita on haluttu viime vuosina kartoittaa ketterän kehittämistä varten. Merkittävimmäksi haastealueeksi ilmaantuvuuden ja tuottavuusvaikutusten perusteella on todettu ketterää käyttävän ryhmän hallinta (\textit{team management}). 

Tämä tutkielma keskittyy ja vastaa seuraavanlaisiin tutkimuskysymyksiin: 1. Mitä hallinnallisia haasteita liittyy ketterään ryhmään? 2. Mitä ratkaisumalleja on olemassa ketterän ryhmän hallinnallisiin haasteisiin?

Yleisimmiksi haastealueiksi muodostuivat koordinaatio ja kokoonpanolliset haasteet. Koordinaatiohaasteita ovat ketterän käyttöönotto, kommunikointi, useamman ryhmän välinen yhteistoiminta sekä ryhmäorientaatio. Kokoonpanollisia haasteita ovat ryhmän suunnitteluvalinnat sekä ryhmäläisten vaihtuvuus. Ratkaisuina on esitetty konsultoinnin hakeminen ammattilaiselta, ryhmälähtöinen ongelmanratkaisu, epävirallisten kommunikointitapojen ja lähitapaamisten suosiminen, virallisen kommunikoinnin lisääminen, selkeä roolijako, paikallisryhmien muodostaminen, jatkuvan kommunikoinnin mahdollistavan työkulttuurin suosiminen, luottamuksen lisääminen, lisäkommunikointi asiakkaan kanssa, ketterän ryhmän kyky mukautua tilanteisiin ja sopiva ohjaus organisaatiojohdon toimesta.

Tutkielma toteaa tarpeen lisätutkimuksille ratkaisumallien osalta. Tarve perustuu ratkaisumalleja tarjoavien lähteiden vähäiseen lukumäärään.

\end{abstract}
