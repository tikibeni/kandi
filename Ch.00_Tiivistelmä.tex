\begin{abstract}

[Kuin johdanto, mutta tiiviimpi. 100-250 samaa: Tutkimuksen tausta, tutkimuskysymykset, tulokset ja vaikuttavuus/yhteenveto]

\textbf{Tausta:}

Ketterä ohjelmistokehitys on kehitetty korjaamaan perinteisten tuotantomenetelmien ja -mallien ongelmakohtia. Vaikka ketterä manifesti korjasi ketterää edeltävän iteratiivisen mallin aatteelliset ongelmat, ketterä sisältää omat haastealueensa, joita on haluttu viime vuosina kartoittaa ketterän kehittämistä varten. Merkittävimmäksi haastealueeksi ilmaantuvuuden perusteella on todettu ketterää käyttävän ryhmän hallinta (\textit{team management}). 

\textbf{Tutkimuskysymykset:}

Tämä kirjallisuuskatsauksellinen tutkielma keskittyy ja vastaa seuraavanlaisiin tutkimuskysymyksiin: 1. Mitä hallinnallisia haasteita liittyy ketterään ryhmään? 2. Mitä ratkaisumalleja on olemassa ketterän ryhmän hallinnallisiin haasteisiin?

\textbf{Tulokset:}


\textbf{Yhteenveto:}



\end{abstract}
