\chapter{Koordinaatio}

Ketterän ryhmän hallinnallisiin haasteisiin vaikuttaa voimakkaasti niin ryhmän sisäinen kuin organisaation puolelta tuleva koordinointi. Koordinaatio kattaa ryhmään tai ryhmiin liittyvät yhteistoimintaa koskevat asiat aina ketterän käyttöönotosta lähtien.

\section{Ketterän käyttöönotto}

Ketterän koordinointiin liittyy niin pienillä kuin suuremmillakin ryhmäkokonaisuuksilla lukuisia haasteita. Juurihaasteena koordinoinnissa toimii puutteellisuus ketterien menetelmien ja periaatteiden ymmärrykseen, joka voi ilmentyä esimerkiksi ketterän menetelmän käyttöönoton yhteydessä olla seuraamusta motivaation puutteesta käyttöönottoon \cite{GREGORY201692}. Käyttöönottoon on pääteltävissä tilanne, jossa organisaatio tiedostaa ketterän tuottavan parempaa kilpailukykyä, mutta on harjoittanut vesiputousmallia \cite{MCKNIGHT2014168} pidemmän aikaa. Organisaatiossa pidemmän aikaa vesiputousaatteella työskennellyt saattaa olla haluton muutokseen, joka itsessään on yksi haasteista. Sivussa motivaatiokysymyksestä mikäli organisaatio tai kehitysryhmät eivät ymmärrä ketterän hyötyjä, siihen liittyviä riskejä tai varsinaisia toteutusmenetelmiä on odotettavissa kivinen käyttöönotto, joka ei välttämättä mene maaliin asti. 

Tutkimuksessaan Gregory et al. \cite{GREGORY201692} toteavat useita haasteita liittyen ketterän toteuttamiseen ja käyttöönottoon epä-ketterän ympäristön ja organisaation kontekstissa. Eräänä rajoittavana tekijänä käyttöönottoon liittyen todettiin ketterän väärinkäyttö tai -ymmärrys. Esimerkkinä väärinkäytöstä toimii tutkimuksessa esiin tullut tilanne, jossa organisaation johto on pakottanut ketterän ryhmän työskentelemään massiivisen määrän ylitöitä perustellen sen olevan osaa ketterää aiheuttaen kehittäjille työuupumusta. Toisena esimerkkinä väärinkäytöstä ilmeni tilanne, jossa ketterän käyttöönoton jälkeen johto on mikromanageerannut ketterää ryhmää ainakin siihen saakka, kunnes nähtävää tulosta on alkanut muodostumaan. Jatkuvan mikromanageroinnin seurauksena on nähtävissä varsinaisen ketteryyden ja sen sisältävän luovuuden väheneminen.

