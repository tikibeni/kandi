\chapter{Koordinaatio}

Koordinaatiota voidaan tarkastella suurimpana vaikuttavana alueena ketterää ohjelmistokehitystä noudattavassa ryhmässä. Aihetta voidaan ajatella koko toimintaketjun sitovana liimana, sillä kunkin esitettävän aihealueen voidaan tulkita kuuluvan koordinaation alle. Tämä tutkielma jaottelee koordinaation ketterän käyttöönottoon ja kommunikaatioon liittyviin seikkoihin. Koordinaatio aiheena kattaa kaiken, joka voidaan tulkita vaikuttavan kehitysryhmän sisäiseen tai ulkoiseen yhteistoimintaan. Ryhmän sisäiseksi koordinaatioksi tulkitaan tässä kontekstissa ryhmäläisten välinen interaktio, kun taas ulkoiseksi koordinaatioksi luetaan esimerkiksi organisaation puolelta tuleva ohjeistaminen (tai ohjaus), mahdollisten muiden saman projektin parissa työskentelevien ryhmien kanssa yhteistyöskentely ja asiakkaan kanssa toimiminen. Esimerkkitapaus ulkoisesta koordinaatiosta on projekti, joka toteutetaan maailmanlaajuisesti hajautetulla ketteränä ohjelmistokehityksenä \cite{ALZOUBI201622}. Esitettävät haasteet voidaan tulkita vaikuttavan voimakkaammin niin laajemmissa ja maailmanlaajuisesti hajautetuissa ketterän kehityksen ryhmissä verrattuna niin sanotusti paikalliseen ryhmään. Maailmanlaajuisesti hajautettu ketterä kehitys tarkoittaa sitä, että prosessi koostuu ryhmistä tai ryhmäläisistä, jotka sijaitsevat maantieteellisesti kaukana toisistaan. Paikallinen ketterä kehitys tarkoittaa tässä kontekstissa vastakohtaa. Koordinaation ollessa varsin laaja käsite, siihen on yhdistettävissä lukuisia mahdollisia haasteita aina yhteistoiminnan käyttöönotosta ylläpitoon.

\section{Ketterän käyttöönotto}

Ensimmäisenä haasteena ketterän koordinointiin liittyen on ketterän menetelmän käyttöönotto ja siispä ketteriä periaatteita noudattavan ryhmän luonti. Eräänä haasteena käyttöönottoon liittyen on todettu tilanne, jossa perinteisiä malleja, kuten vesiputousta harjoittanut organisaatio haluaa ketteryyttä pohjautuen tietoon, että ketterä tuottaa parempaa kilpailukykyä \cite{MCKNIGHT2014168}. Haasteita tässä tapauksessa voi tuottaa organisaation ymmärrystaso ketteristä periaatteista ja niiden toteutustavoista. Puutteellinen ymmärrys voi olla seuraamusta motivaation puutteesta käyttöönottoon, joka juontaa juurensa menetelmätottumuksiin \cite{GREGORY201692}. Mikäli organisaatiossa ei ymmärretä ketteryyteen liittyviä hyötyjä, siihen liittyviä riskejä tai varsinaisia toteutusmenetelmiä on odotettavissa kivinen tie käyttöönoton suhteen, mikä ei välttämättä mene maaliin asti.

Tutkimuksessaan Gregory et al. \cite{GREGORY201692} toteavat useita haasteita liittyen ketterän toteuttamiseen ja käyttöönottoon perinteisen ympäristön ja organisaation kontekstissa. Eräänä rajoittavana tekijänä käyttöönottoon liittyen todettiin ketterän väärinkäyttö, joka johtuu väärinymmärryksestä. Tutkimuksessa ilmeni tilanne väärinkäytöstä, missä organisaation johto on teettänyt niin sanotun ketterän ryhmän työskentelemään massiivisen määrän ylitöitä perustellen sen olevan osana ketterää menetelmää. Skenaariossa on nähtävissä kehittäjien loppuunpalaminen sekä organisaation paluu perinteisten menetelmien käyttöön, kun uusi menetelmä ei olekaan tuottanut odotettua tulosta. Toisena väärinkäyttöön liittyvänä skenaariona ilmeni tilanne, jossa jonkin ketterän menetelmän käyttöönoton jälkeen organisaation johto on alkanut mikromanageroimaan kehitysryhmää ainakin siihen saakka, kunnes nähtävää tulosta on alkanut muodostumaan. Jatkuvan mikromanageroinnin seurauksena on nähtävissä varsinaisen ketteryyden ja sen sisältävän luovuuden väheneminen.

Kaiken kaikkiaan ketterän käyttöönottoon liittyvät haasteet voidaan minimoida tai jopa välttää sillä, että organisaatiossa perehdytään tarvittavalla tasolla uusiin menetelmiin [\textbf{LÄHDE}]. Käyttöönottoa voidaan verrata muutokseen työkulttuurissa, joka itsessään ilmaisee tehokkaasti tilanteen vaativuuden.

\section{Kommunikaatio}

Kommunikaation voidaan ajatella olevan ketterän ryhmän koordinaation mahdollistava työkalu. Ilman kommunikaatiota ei olisi yhteistoimintaa. Kommunikaatio on siispä äärimmäisen tärkeää saada toimimaan alusta alkaen. Kommunikaatioon liittyvät puutteet ovat varma tie haasteisiin. Haasteiden minimoimiseksi olisi olennaista, että ketterä ryhmä kommunikoi ryhmäläistensä kesken ja ryhmän ulkopuolella olevien yhteistyötahojen kanssa. Kommunikaation merkitys korostuu maantieteellisesti hajautettujen ryhmien tai ryhmäläisten osalta, sillä kyseiset tahot eivät lähtökohtaisesti pääse olemaan samassa työskentelytilassa muiden kanssa. Voi olla esimerkiksi tilanne, jossa kukin ketterän ryhmän jäsen sijaitsee eri maassa ja mahdollisesti eri aikavyöhykkeillä. Ketterien periaatteiden korostaessa asiakasyhteistyötä on myös ratkaisevaa, että ryhmä kommunikoi asiakkaan kanssa riittävän selkeästi ja usein.

Tutkimuksessaan Alzoubi et al. \cite{ALZOUBI201622} keskittyvät ketterään liittyvän kommunikaation pohtimiseen ja tuovat esille siihen liittyviä haasteita ja ratkaisuja. Tutkimus itsessään keskittyy maailmanlaajuisesti hajautetun ketterän ohjelmistokehityksen kommunikatiivisiin seikkoihin, mutta niitä voidaan soveltaa paikallisesti toteutettavaan variaatioon. Tutkimuksessa todettiin kommunikaatiohaasteiden johtavan kaikkiin muihin haasteisiin ja että niiden käsittely vaatii 2.5 kertaa enemmän resursseja kuin paikallisessa variaatiossa. Alzoubi et al. jakavat kommunikaation ketterässä ohjelmistokehityksessä henkilökohtaiseen, vuorovaikutteiseen ja vertais-orientoituneeseen ja toteavat sen nojaavan voimakkaasti interaktioon, joka tapahtuu kasvotusten niin ryhmäläisten kuin ryhmän ja asiakkaan välillä. Kasvotusten kommunikoinnin on todettu johtavan epäformaalimpaan tapaan kommunikoida, minkä vuorostaan on todettu olevan tehokkaampaa.

Alzoubi et al. tutkimuksessa \cite{ALZOUBI201622} todettiin, että kommunikaatiohaasteita maailmanlaajuisen ketterän kehityksen kontekstissa aiheuttavat muun muassa (esiintyvyysjärjestyksessä) etäisyys, organisaatiotekijät, ihmistekijät ja ryhmäkonfigurointi. Etäisyyden todettiin olevan yleisin haasteiden lähde. Sen todettiin hankaloittavan koordinaatiota esimerkiksi viiveinä kommunikoinnissa, pidempinä kokouksina, hankaluutena löytää yhteisiä työaikoja ja luottamuksen puutteena. 