\chapter{Suunnittelu}

Suunnittelun merkitys nykyaikaisessa ohjelmistokehityksessä on vähentynyt perinteisiin malleihin, kuten vesiputousmalliin nähden. Toisaalta puutteet suunnittelussa on kuitenkin yksi yleisimmistä ketterää toteuttavan ryhmän hallinnallisten haasteiden lähteistä \cite{7872736}, minkä myötä sitä ei kuitenkaan saa unohtaa. Beck et al. \cite{beck2001agile} määrittävät yhtenä ketteristä periaatteista, että muutokseen reagointi on tärkeämpää kuin suunnitelmassa pysyminen. Suunnitteluun liittyvät haasteet voidaan jakaa karkeasti ryhmäkokoonpanon suunnitteluun, joka itsessään jakautuu de Melo et al. \cite{DEOMELO2013412} mukaisesti ryhmän sisäisiin ja ulkoisiin tekijöihin.

\section{Ryhmäkokoonpano}

Tutkimuksessaan de Melo et al. \cite{DEOMELO2013412} toteavat ketterän ryhmän hallinnan olevan suurin vaikuttava tekijä tuottavuuteen. Tutkimus kartoitti merkittävimmät tuottavuuteen vaikuttavat tekijät, minkä myötä ne rajattiin ketterän ryhmän sisäisiin ja ulkoisiin haasteisiin. Ryhmän sisäisistä haasteista merkittävimmiksi korostuivat ryhmän suunnitteluvalinnat sekä jäsenten vaihtuvuus, kun taas ulkoisista haasteista merkittävimmäksi muodostui ryhmien välinen koordinaatio.

[Katso löytyisikö seuraavaan kohtaan jotain lähteitä tukemaan vielä olettamusväitteitä, jotta olisi suorempaa viitettä]

\subsection{Suunnitteluvalinnat}

Ryhmän suunnitteluvalinnat koostuvat \cite{DEOMELO2013412} tutkimuksessa ryhmäkoosta, ryhmäläisten taidoista, ryhmäläisten keskinäisestä sijainnista sekä ryhmäläisten jaosta. Tutkimuksesta on tulkittavissa, että haasteita kokoonpanoon tuottaa tässä kontekstissa muun muassa seuraavanlaiset tilanteet. Kokoonpanossa, jossa valtaosa ryhmäläisistä olisi osa-aikaisia työntekijöitä keskinäinen keskittyminen projektiin voisi olla heikkoa perustuen tutkimuksen toteamukseen tapauksesta, jossa täyspäiväjäsenillä on todetusti parempi keskittyminen. [Lisää järkevästi muotoiltuja kohtia tähän liittyen]

\subsection{Vaihtuvuus}

Ryhmän jäsenten vaihtuvuus on toinen ryhmän sisäistä haasteista \cite{DEOMELO2013412}. Vaihtuvuudeksi de Melo et al. määrittävät sekä aiemman ryhmäläisen lähtönä ryhmästä että uuden kehittäjän liittymisenä ryhmään. Vaihtuvuus aiheuttaa organisaatiolle monenlaisia kustannuksia, kuten irtisanoutumiseen liittyvät, rekrytointiin sekä perehdyttämiseen liittyviä menoja. Lisäksi kehitysryhmä työskentelee vaihtuvuuden aikana astetta pienemmällä teholla. Tutkimuksessaan de Melo et al. määrittävät vaihtuvuuden tekijöiksi palkkatason sekä kehittäjän sopeutumattomuutena ryhmään. Sopeutumattomuuden on arvioitu johtuvan muun muassa oma-aloitteisuuden ja sitoutumisen puutteesta, epäsopivasta personaallisuudesta sekä ryhmän sisäisistä konflikteista. Viimeisestä on tulkittavissa, että haasteet kommunikoinnissa voivat aiheuttaa suunnitelmallisia haasteita.

Vaikka vaihtuvuuden on todettu aiheuttavan suunnitelmallisia haasteita, niin sen on todettu myös vaikuttavan positiviisesti uusien ryhmäläisten kautta \cite{DEOMELO2013412}. On todettu, että uudet ryhmäläiset saattavat tuoda uusia ideoita, ratkaisuja ja yleisesti ottaen energiaa ryhmään. Toisaalta vaihtuvuuden negatiivisia vaikutuksia on lukumäärällisesti enemmän ja ne voidaan jakaa pidentyneeksi tuotantoajaksi lyhyellä aikavälillä, oleellisen tiedon menettämisenä poistumisen yhteydessä sekä lisäkustannuksina. 
