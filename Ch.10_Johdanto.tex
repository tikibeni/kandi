\chapter{Johdanto\label{intro}}

Ketterä ohjelmistokehitys (tai lyhyemmin \textit{ketterä}) on suosittu kehitysaate nykypäivän ohjelmistokehityksessä, minkä avulla ohjelmistoja voidaan rakentaa ja muokata tehokkaasti asiakkaan tarpeiden ympärille. Ketterän suosio perustuu periaatteisiin, joiden myötä muun muassa ohjelmistoprojektien onnistuvuusprosentin ja asiakastyytyväiseyyden on todettu nousevan \cite{9533020}. Ketterää toteutetaan niin sanotuilla ketterillä menetelmillä, joista yleisimpien joukkoon lukeutuvat esimerkiksi Scrum \cite{SCRUMORG} ja XP \cite{XPORG}. Ketterää on haluttu kehittää entisestään tutkimalla ja kartoittamalla yleisimpiä haasteita ja niiden syitä \cite{7872736}.

Ketterä ohjelmistokehitys ja sen menetelmät kehitettiin korjaamaan sitä edeltävien mallien, kuten niin sanotun vesiputousmallin omaavia ongelmakohtia \cite{beck2001agile}. Vuonna 1970 esitetyn vesiputousmallin ideana on tuottaa ohjelmistoa täsmällisessä järjestyksessä yksi vaihe kerrallaan \cite{10.5555/41765.41801}. Nimensä mukaisesti mallilla tuotettavat projektit etenevät ylhäältä alas seuraavanlaisessa järjestyksessä: vaatimusmäärittely, suunnittelu, toteutus, testaus, tuotanto ja ylläpito. Malli siispä olettaa, että esimerkiksi ennen ohjelmiston suunnittelua voitaisiin kartoittaa kaikki asiakkaalta tulleet vaatimukset ohjelmistoa kohtaan puhumattakaan siitä, että ennen pienintäkään toteutusta ohjelmisto suunniteltaisiin alusta loppuun. Tässä päästään vesiputousmallin ongelmiin. Mallia voisi ajatella pienenä korttitalona, joka koostaa ohjelmiston. Ensimmäiset vaiheet, vaatimusmäärittely ja suunnittelu ovat alimmassa kerroksessa ja loput vaiheet muodostavat ylemmät kerrokset siten, että ylläpito on ylimpänä. Kuvitellaan, että palikkaan, joka koostaa tässä tapauksessa asiakkaan tarvitsisi tehdä muutoksia jossain jälkimmäisessä vaiheessa. Tästä seuraisi korttitalon kaatuminen mallin sääntöjä ajatellen. Vesiputousmallin yksi suurimmista ongelmista on siis, että se olettaa kunkin vaiheen olevan konklusiivinen esimerkiksi siten, että asiakas tietäisi alusta alkaen täsmälleen mitä haluaa. Malli ei jätä varaa projektin aikaisille muutoksille.

Vesiputousmallin ongelmat tiedostettiin, minkä myötä 90-luvulla yleistyi dynaamisempi malli, iteratiivinen (joka on samalla myös inkrementaalinen) ohjelmistokehitys \cite{1204375}. Iteratiivinen kehitys perustuu nimensä mukaisesti kehitysprosessin iteraatioihin pilkkomiseen. Yhdessä iteraatiossa määritellään, suunnitellaan, toteutetaan ja testataan ohjelmistoa, minkä jälkeen versiota esitellään asiakkaalle. Tätä sykliä jatketaan, kunnes ohjelmisto on valmis ja se viedään tuotantoon. Asiakaskeskustelun huomiot kirjataan ylös ja ohjelmistoa rakennetaan seuraavassa iteraatiossa kirjatut asiat huomioon ottaen. Menetelmän rakenteesta huomaa, että se on jo huomattavasti lähempänä nykypäivän menetelmiä ja muistuttaa esimerkiksi Scrum-menetelmässä \cite{SCRUMORG} esiintyviä sprinttejä. Iteratiivisen mallin ongelmakohtana on sen periaatteellinen tausta \cite{1204375}. Vaikka malli korjaa prosessitasolla vesiputousmallin ongelmakohtia, niin se silti omaa vesiputousmallin periaatteita. Tarkka suunnittelu ja dokumentaatio ovat edelleen hyvin korostetussa asemassa ja työntekijästä on tehty minimaalisesti projektiin vaikuttava tekijä, joka on tarvittaessa helposti korvattavissa. Iteratiiviseen malliin liittyykin yleinen kritiikin kohta, jonka mukaan siinä toteutettaisiin vesiputousmallia, mutta vain pilkottuna iteraatioihin \textbf{[Lähde?]}. 

Iteratiivisen ohjelmistokehityksen ristiriidat otettiin käsittelyyn vuonna 2001, kun ketterien periaatteiden aatteelliset kehittäjät kokoontuivat yhteen pohtimaan kokonaisuutta. Tuloksena tapaamisesta syntyi niin sanottu ketterä manifesti \cite{beck2001agile}. Manifesti sisältää ketterät periaatteet, joiden pohjalta nykypäivän ketterät menetelmät on kehitetty. Periaatteissa määritellään muun muassa, että yksilöt ja vuorovaikutus ovat tärkeämpiä kuin prosessit ja työkalut sekä muutoksiin reagoidaan sen sijaan, että seurattaisiin täsmällisesti ennalta määriteltyä suunnitelmaa. Ketterä ohjelmistokehitys on ollut nyt vallitsevana mallina noin 20 vuotta, mutta se ei poissulje sitä, etteikö se sisältäisi ongelmakohtia tai haastealueita. Mallin pysyvyyden pohjalta voidaan todeta, että aatteen tasolla ketterä toimii paremmin kuin vesiputous- ja iteratiivinen malli. Aatteen toimiessa hyvin sen haasteiden kartoittaminen ja menetelmien kehittäminen siirtyy tarkastelun osalta yksilöön ja kehitysryhmiin.

Tutkimuksessaan Fitriani et al. \cite{7872736} totesivat yleisimmäksi haastealueeksi ketterää käyttävässä ryhmässä hallinnalliset (\textit{team management}) haasteet jaetulla ilmaantuvuudella hajautettuihin ryhmiin liittyvissä haasteissa, jotka itsessään liittyvät vahvasti ensimmäiseen. Tutkimus julkaistiin vuoden 2016 ICACSIS-konferenssissa tavoitteenaan tuoda esille ketterään liittyviä yleisiä haasteita. Tutkimuksessa koottiin yhteen 20 eri haasteita käsittelevää julkaisua, joista ilmeni kaiken kaikkiaan 30 haastetta. Yhdeksässä tutkimuksista käsiteltiin ketterään ohjelmistokehitysryhmään liittyviä hallinnallisia haasteita, joihin lukeutuu kaikki koordinointiin, suunnitteluun, ryhmäläisten käytökseen sekä ryhmän käytänteisiin liittyvät ilmiöt. Ketterää käyttävän ryhmän hallinnalliset haasteet ovat merkittäviä johtuen ketteristä periaatteista, joiden myötä muun muassa yksilö ja vuorovaikutus asetetaan tärkeämmäksi suhteutettuna prosesseihin ja työkaluihin \cite{beck2001agile}. Periaatteen myötä ryhmä on lähtökohtaisesti toiminnassaan autonominen, jonka myötä sen hallinta ja suunnan ohjaus tapahtuu sisältäpäin. Hallinnallisuuteen liittyvät haasteet ovat myös merkittäviä siksi, että aihealueen on todettu olevan suurin vaikuttava tekijä kehitykselliseen tuottavuuteen \cite{DEOMELO2013412}.

Ketteryyteen liittyvään autonomisuuteen liittyy sekä vapautta muotoilla sopivat työskentelytavat että riskejä pahimmassa tapauksessa syöstä projekti raiteiltaan. Ilman ammattitaitoista ja ketteriin periaatteisiin tutustunutta ryhmää, ryhmän projektin ohjaus ja hallinta voivat olla hyvinkin haastavia \cite{7872736}. Vaikka ketterässä on monia hyviä puolia niin ryhmädynamiikan kuin asiakkaan kannalta, ketterän ryhmän toimintaan liittyy lukuisia mahdollisia haasteita. Haasteita ilmenee esimerkiksi tapauksissa, joissa ketterää yritetään käyttää tai toteuttaa puutteellisesti. Yksi merkittävimmistä haasteiden lähteistä on asetelma, jossa ketterää toteutetaan ymmärtämättä sen ydinperiaatteita. Toinen yleinen lähde haasteille on tilanne, jossa ketterää menetelmää käytetään puuttellisesti, jonka myötä joistain käytetyn menetelmän käytänteistä luovutaan tai jätetään vähemmälle. Esimerkkinä tilanteesta toimii niin sanottu ScrumBut \cite{SCRUMBUT}, jossa ketterä ryhmä on ottanut käyttöön Scrumin, mutta ei toteuta sitä käytänteiltään täysimittaisesti esimerkiksi viikottaisten tapaamisten tai retrospektiivien suhteen.

Esitetyt skenaariot johtavat suuremmalla todennäköisyydellä ongelmiin ketterässä ryhmässä ja sen toteuttamassa projektissa, minkä myötä projektin ja ketterän ryhmän hallinnasta voi tulla hyvinkin haastavaa. Haasteita käsiteltäessä on syytä huomioida ketterien ryhmien eri ryhmäkoot. Haasteet voivat esiintyä minkä kokoisessa ryhmässä tahansa niiden perustuessa samaan aatteeseen, mutta niiden voidaan olettaa muodostavan suurempaa riskiä ja potentiaalista vahinkoa suuremmissa variaatioissa, kuten maailmanlaajuisesti hajautetussa ketterässä kehityksessä \cite{ALZOUBI201622}. Tulevat kappaleet toimivat yleisimpien haastealueiden ja niihin liittyvien ilmiöiden esittelijöinä ja käsittelijöinä. On huomioimisen arvoista, että haasteet voivat mennä useamman kuin yhden kategorian alle, minkä myötä haastealueisiin jako ei ole täysin poissulkevaa. Tässä tutkielmassa keskitytään pohtimaan seuraavia kysymyksiä: \begin{enumerate}
    \item Mitä hallinnallisia haasteita liittyy ketterään ryhmään?
    \item Mitä ratkaisumalleja on olemassa ketterän ryhmän hallinnallisiin haasteisiin?
\end{enumerate}
