\chapter{Johdanto\label{intro}}

Ketterä ohjelmistokehitys (jatkossa \textit{ketterä}) on suosittu kehitysaate nykypäivän ohjelmistokehityksessä, minkä avulla ohjelmistoja voidaan rakentaa ja muokata tehokkaasti asiakkaan tarpeiden ympärille. Ketterän suosio perustuu periaatteisiin, joiden myötä ohjelmistoprojektien onnistuvuusprosentin ja asiakastyytyväiseyyden on todettu nousevan \cite{9533020}. Ketterää toteutetaan niin sanotuilla ketterillä menetelmillä, joista yleisimpien joukkoon lukeutuvat esimerkiksi Scrum \cite{SCRUMORG} ja XP \cite{XPORG}. Ketterää on haluttu kehittää entisestään tutkimalla ja kartoittamalla yleisimpiä haasteita ja niiden syitä \cite{7872736}.

Tutkimuksessaan Fitriani et al. \cite{7872736} totesivat yleisimmäksi haastealueeksi ketterää noudattavassa ryhmässä (jatkossa \textit{ketterä ryhmä}) hallinnalliset haasteet jaetulla ilmaantuvuudella hajautettuihin ryhmiin liittyvissä haasteissa, mitkä itsessään liittyvät vahvasti ensimmäiseen. Tutkimus julkaistiin vuoden 2016 ICACSIS-konferenssissa tavoitteenaan tuoda esille ketterään liittyviä yleisiä haasteita. Tutkimuksessa koottiin yhteen 20 eri haasteita käsittelevää tutkimusta, joista ilmeni kaiken kaikkiaan 30 haastetta. Yhdeksässä tutkimuksista käsiteltiin ketterään ohjelmistokehitysryhmään liittyviä hallinnallisia haasteita, joita ovat muun muassa suunnitteluun, koordinointiin, käytökseen sekä ryhmän käytänteisiin liittyvät huomiot. Ketterän ryhmän hallinnalliset haasteet ovat merkittäviä johtuen ketteristä periaatteista, joiden myötä muun muassa yksilö ja vuorovaikutus asetetaan tärkeämmäksi suhteutettuna prosesseihin ja työkaluihin \cite{beck2001agile}. Periaatteen myötä ketterä ryhmä on lähtökohtaisesti toiminnassaan autonominen, jonka myötä ryhmän hallinta ja suunnan ohjaus tapahtuu ryhmän toimesta.

Ketteryyteen liittyvään autonomisuuteen liittyy sekä vapautta muotoilla sopivat työskentelytavat että riskejä syöstä projekti pahimmassa tapauksessa raiteiltaan. Ilman ammattitaitoista ja ketteriin periaatteisiin tutustunutta ryhmää ryhmän projektin ohjaus ja ryhmänhallinta voivat olla hyvinkin haastavia \cite{7872736}.

Tässä tutkielmassa keskitytään pohtimaan seuraavia kysymyksiä: \begin{enumerate}
    \item Mitä hallinnallisia haasteita liittyy ketterään ryhmään?
    \item Mitä ratkaisumalleja on olemassa ketterän ryhmän hallinnallisiin haasteisiin?
\end{enumerate}

Tämän tutkielman ensimmäisessä puoliskossa pureudutaan ketterän ryhmän hallintaan liittyviin haastealueisiin ja jälkimmäisellä puoliskolla käsitellään esitettyihin haastealuesiin liittyviä mahdollisia ratkaisumalleja.
