\chapter{Johdanto\label{intro}}

Ketterä ohjelmistokehitys (jatkossa \textit{ketterä}) on suosittu kehitysaate nykypäivän ohjelmistokehityksessä, minkä avulla ohjelmistoja voidaan rakentaa ja muokata tehokkaasti asiakkaan tarpeiden ympärille. Ketterän suosio perustuu periaatteisiin, joiden myötä ohjelmistoprojektien onnistuvuusprosentin ja asiakastyytyväiseyyden on todettu nousevan \cite{9533020}. Ketterää toteutetaan niin sanotuilla ketterillä menetelmillä, joista yleisimpien joukkoon lukeutuvat esimerkiksi Scrum \cite{SCRUMORG} ja XP \cite{XPORG}. Ketterää on haluttu kehittää entisestään tutkimalla ja kartoittamalla yleisimpiä haasteita ja niiden syitä \cite{7872736}.

Tutkimuksessaan Fitriani et al. \cite{7872736} totesivat yleisimmäksi haastealueeksi ketterää käyttävässä ryhmässä (jatkossa \textit{ketterä ryhmä}) hallinnalliset haasteet jaetulla ilmaantuvuudella hajautettuihin ryhmiin liittyvissä haasteissa, mitkä itsessään liittyvät vahvasti ensimmäiseen. Tutkimus julkaistiin vuoden 2016 ICACSIS-konferenssissa tavoitteenaan tuoda esille ketterään liittyviä yleisiä haasteita. Tutkimuksessa koottiin yhteen 20 eri haasteita käsittelevää tutkimusta, joista ilmeni kaiken kaikkiaan 30 haastetta. Yhdeksässä tutkimuksista käsiteltiin ketterään ohjelmistokehitysryhmään liittyviä hallinnallisia haasteita, joita ovat muun muassa suunnitteluun, koordinointiin, käytökseen sekä ryhmän käytänteisiin liittyvät huomiot. Ketterän ryhmän hallinnalliset haasteet ovat merkittäviä johtuen ketteristä periaatteista, joiden myötä muun muassa yksilö ja vuorovaikutus asetetaan tärkeämmäksi suhteutettuna prosesseihin ja työkaluihin \cite{beck2001agile}. Periaatteen myötä ketterä ryhmä on lähtökohtaisesti toiminnassaan autonominen, jonka myötä ryhmän hallinta ja suunnan ohjaus tapahtuu ryhmän toimesta. Hallinnallisuuteen liittyvät haasteet ovat myös merkittäviä siksi, että aihealueen on todettu olevan suurin vaikuttava tekijä tuottavuuteen \cite{DEOMELO2013412}.

Ketteryyteen liittyvään autonomisuuteen liittyy sekä vapautta muotoilla sopivat työskentelytavat että riskejä syöstä projekti pahimmassa tapauksessa raiteiltaan. Ilman ammattitaitoista ja ketteriin periaatteisiin tutustunutta ryhmää ryhmän projektin ohjaus ja ryhmänhallinta voivat olla hyvinkin haastavia \cite{7872736}. Vaikka ketterässä on monia hyviä puolia niin ryhmädynamiikan kuin asiakkaan kannalta, ketterän ryhmän toimintaan liittyy lukuisia mahdollisia haasteita. Haasteita ilmenee esimerkiksi tapauksissa, joissa ketterää yritetään käyttää tai toteuttaa puutteellisesti. Yksi merkittävimmistä haasteiden lähteistä on asetelma, jossa ketterää toteutetaan ymmärtämättä sen ydinperiaatteita. Toinen yleinen lähde haasteille on tilanne, jossa ketterää menetelmää käytetään puuttellisesti, minkä myötä joistain käytetyn menetelmän käytänteistä luovutaan tai jätetään vähemmälle. Esimerkkinä tilanteesta toimii niin sanottu ScrumBut \cite{SCRUMBUT}, jossa ketterä ryhmä on ottanut käyttöön Scrumin, mutta eivät toteuta sitä käytänteiltään täysimittaisesti esimerkiksi viikottaisten tapaamisten tai retrospektiivien suhteen.

Edellä pohditut skenaariot johtavat suuremmalla todennäköisyydellä haasteisiin ketterässä ryhmässä ja projektissa, minkä myötä projektin ja ketterän ryhmän hallinnasta voi tulla hyvinkin haastavaa. Haasteita käsiteltäessä on syytä huomioida ketterien ryhmien eri ryhmäkoot. Haasteet voivat esiintyä minkä kokoisessa ryhmässä tahansa niiden perustuessa samaan aatteeseen, mutta niiden voidaan olettaa muodostavan suurempaa riskiä ja potentiaalista vahinkoa suuremmissa variaatioissa, kuten maailmanlaajuisesti hajautetussa ketterässä kehityksessä \cite{ALZOUBI201622}. Tulevat kappaleet toimivat yleisimpien haastealueiden ja niihin liittyvien haasteiden esittelijöinä ja käsittelijöinä. Tietenkin on huomioimisen arvoista, että haasteet voivat mennä useamman kuin yhden kategorian alle, minkä myötä haastealueisiin jako ei ole täysin poissulkevaa. Tässä tutkielmassa keskitytään pohtimaan seuraavia kysymyksiä: \begin{enumerate}
    \item Mitä hallinnallisia haasteita liittyy ketterään ryhmään?
    \item Mitä ratkaisumalleja on olemassa ketterän ryhmän hallinnallisiin haasteisiin?
\end{enumerate}
